\documentclass[nooutcomes]{ximera}

\title{On the Infinitude of Primes}

\begin{document}
\begin{abstract}
    
\end{abstract}
\maketitle



The great theorem of this chapter is, essentially, that there are infinitely many primes.  In our readings, we'll see Euclid's proof of this fact as well as another proof by a mathematician named Hillel Furstenberg. Furstenberg is probably most famous for his contributions to an area of mathematics called ``ergodic theory'', in which we study moving systems.


\section{Readings}
First reading: Dunham, Chapter 3, pages 73-83

Second reading: \link[On Furstenburg's Proof of the Infinitude of Primes ]{http://www.jstor.org.proxy.lib.ohio-state.edu/stable/40391095}. (Library login required off-campus.)



\section{Questions}

\begin{question}
What is the example given of an arithmetic progression? $2 + \answer[given]{7} \mathbb{Z}$
\end{question}

\begin{question}
To which branch of mathematics is Furstenberg's proof method most related?  In other words, what makes his approach different from Euclid's?
\begin{multipleChoice}
\choice{Number Theory}
\choice {Calculus}
\choice [correct]{Topology}
\choice {Ergodic Theory}
\end{multipleChoice}
\end{question}


\begin{question}
What are the most important points from this reading?
\begin{freeResponse}
\end{freeResponse}

\end{question}




\end{document}
