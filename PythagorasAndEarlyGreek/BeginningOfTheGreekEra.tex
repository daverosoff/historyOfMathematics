\documentclass{ximera}

\title{The Beginning of the Greek Era}

\begin{document}
\begin{abstract}
\end{abstract}
\maketitle

Now that we have looked at the mathematics done by the ancient Egyptians and Babylonians, we turn our attention to the Greeks.  Here, we find the first evidence of what we might recognize today as ``mathematics'': statements about objects and their properties, followed by proofs of these statements.  We will begin by considering two of the most well-known Greek mathematicians: Thales and Pythagoras.




\section{Readings}

First Reading: \link[Thales: Our Founder?]{http://www.jstor.org/stable/3615512}

Second Reading: \link[Pythagoras biography]{http://www-history.mcs.st-and.ac.uk/Biographies/Pythagoras.html}

Third Reading: \link[A Brief History of Numbers]{http://ebookcentral.proquest.com.proxy.lib.ohio-state.edu/lib/ohiostate-ebooks/detail.action?docID=2084891}
\begin{itemize}
\item Read Section 2.2, pages 28 - 30. (The earlier parts of Section 2.2 are a review of Egypt and Babylon if you'd like to read them.)
\item Read Chapter 3, pages 31 - 42
\end{itemize}



\section{Questions}

\begin{question}
How many theorems was Thales said to have proven? $\answer[given]{5}$
\end{question}


\begin{question}
What is Pythagoras primarily considered?
\begin{multipleChoice}
\choice[correct]{A philosopher.}
\choice{A mathematician.}
\choice{A traveler.}
\choice{An astronomer.}
\end{multipleChoice}
\end{question}






\begin{question}
What are the most important points of this reading?
\begin{freeResponse}
\end{freeResponse}
\end{question}



\end{document}