\documentclass[nooutcomes]{ximera}

\title{The Riemann Hypothesis}

\begin{document}
\begin{abstract}
    
\end{abstract}
\maketitle

In Chapter 9, we meet perhaps the most prolific mathematician of all time: Euler.  Dunham has written a lot in his career about Euler, and so we have two chapters to study his work.  In this first chapter, our Great Theorem will investigate how Euler succeeded where Leibniz and Bernoulli failed to find the sum of $\sum \frac{1}{n^2}$.  Euler didn't stop adding up series here, though, and has an impressive list of results similar to this one.

Euler's results, however, quickly bring up another question in the form of ``how far can we really go, here?''  It's natural at this point to introduce what's known as the Riemann hypothesis.  Riemann himself, as well as the hypothesis and Euler's work surrounding it are all described in the second reading.  The reading itself appears to be long, but should hopefully be a quick read.  Keep in mind, though, that the mathematician who wrote it is joking about just about everything other than the mathematics!


\section{Readings}
First reading: Dunham, Chapter 9

Second reading: \link[A Friendly Introduction to the Riemann Hypothesis]{http://www.math.jhu.edu/~wright/RH2.pdf}



\section{Questions}

\begin{question}
How many problems are in Hilbert's list in total? $\answer[given]{23}$
\end{question}

\begin{question}
What is the value of $\zeta(-2k)$ for $k$ a positive integer?
\begin{multipleChoice}
\choice{No one knows.}
\choice {The function blows up to infinity there.}
\choice {A purely imaginary number.}
\choice [correct]{$0$}
\end{multipleChoice}
\end{question}


\begin{question}
What are the most important points from this reading?
\begin{freeResponse}
\end{freeResponse}

\end{question}




\end{document}
