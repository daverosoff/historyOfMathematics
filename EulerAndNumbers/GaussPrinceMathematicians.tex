\documentclass[nooutcomes]{ximera}

\title{Gauss, The Prince of Mathematicians}

\begin{document}
\begin{abstract}
    
\end{abstract}
\maketitle



One of the downsides of using a textbook like Dunham's is that we only have time to talk about a small number of mathematicians.  At this point in history, we begin to see an increasing number of names you would recognize from your studies, both in mathematics as well as other subjects.  To put Gauss in the epilogue of a chapter about Euler is, in my opinion, a disservice to this great mathematician.  Some people even consider him to be the greatest mathematician since antiquity - greater even than Euler.  You should, of course, form your own opinions after doing these readings.

We begin with Dunham's biography of Gauss, and then our second reading discusses Gauss's proofs of the Fundamental Theorem of Algebra.  You don't need to be able to give these proofs exactly as they are described, but you should be able to talk about the main ideas.  Finally, an optional third reading is the Mathematics Genealogy Project, beginning with Gauss.  The genealogy project has a page for most modern mathematicians listing both their advisor as well as their advisees.  So, you can click through either forward in time or backward in time to see how mathematicians are connected to one another!


\section{Readings}
First reading: Dunham, Chapter 10, pages 235 - 244

Second reading: \link[Gauss's Proofs of the Fundamental Theorem of Algebra]{math.huji.ac.il/~ehud/MH/Gauss-HarelCain.pdf}

Third reading: \link[Mathematics Genealogy Project]{https://genealogy.math.ndsu.nodak.edu/id.php?id=18231}



\section{Questions}

\begin{question}
How many proofs of the Fundamental Theorem of Algebra did Gauss give? $\answer[given]{4}$
\end{question}

\begin{question}
Which of the following did not attempt a proof of the Fundamental Theorem of Algebra?
\begin{multipleChoice}
\choice{Gauss}
\choice {d'Alembert}
\choice [correct]{Leibniz}
\choice {Euler}
\end{multipleChoice}
\end{question}


\begin{question}
What are the most important points from this reading?
\begin{freeResponse}
\end{freeResponse}

\end{question}




\end{document}
