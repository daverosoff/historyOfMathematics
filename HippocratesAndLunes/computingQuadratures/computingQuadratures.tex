\documentclass[handout]{ximera}

\graphicspath{{./}{thePythagoreanTheorem/}{deMoivreSavesTheDay/}{complexNumbersFromDifferentAngles/}{trianglesOnACone/}{cityGeometry/}{EuclidAndGeometry/}}

\usepackage{gensymb}
\usepackage[margin=1in]{geometry}

%\usepackage{hyperref}


\usepackage{tikz}
\usepackage{tkz-euclide}
\usetkzobj{all}
\tikzstyle geometryDiagrams=[ultra thick,color=blue!50!black]
\newcommand{\tri}{\triangle}
\renewcommand{\l}{\ell}
\renewcommand{\P}{\mathcal{P}}
\newcommand{\R}{\mathbb{R}}
\newcommand{\Q}{\mathbb{Q}}

\newcommand{\Z}{\mathbb Z}
\newcommand{\N}{\mathbb N}
\newcommand{\ph}{\varphi}

\renewcommand{\vec}{\mathbf}
\renewcommand{\d}{\,d}



%% Egyptian symbols

\usepackage{multido}
\newcommand{\egmil}[1]{\multido{\i=1+1}{#1}{\includegraphics[scale=.1]{egyptian/egypt_person.pdf}\hspace{0.5mm}}}
\newcommand{\eghuntho}[1]{\multido{\i=1+1}{#1}{\includegraphics[scale=.1]{egyptian/egypt_fish.pdf}\hspace{0.5mm}}}
\newcommand{\egtentho}[1]{\multido{\i=1+1}{#1}{\includegraphics[scale=.1]{egyptian/egypt_finger.pdf}\hspace{0.5mm}}}
\newcommand{\egtho}[1]{\multido{\i=1+1}{#1}{\includegraphics[scale=.1]{egyptian/egypt_lotus.pdf}\hspace{0.5mm}}}
\newcommand{\eghun}[1]{\multido{\i=1+1}{#1}{\includegraphics[scale=.1]{egyptian/egypt_scroll.pdf}\hspace{0.5mm}}}
\newcommand{\egten}[1]{\multido{\i=1+1}{#1}{\includegraphics[scale=.1]{egyptian/egypt_heel.pdf}\hspace{0.5mm}}}
\newcommand{\egone}[1]{\multido{\i=1+1}{#1}{\includegraphics[scale=.1]{egyptian/egypt_stroke.pdf}\hspace{0.5mm}}}
\newcommand{\egyptify}[7]{
 \multido{\i=1+1}{#1}{\includegraphics[scale=.1]{egyptian/egypt_person.pdf}\hspace{0.5mm}}
 \multido{\i=1+1}{#2}{\includegraphics[scale=.1]{egyptian/egypt_fish.pdf}\hspace{0.5mm}}
 \multido{\i=1+1}{#3}{\includegraphics[scale=.1]{egyptian/egypt_finger.pdf}\hspace{0.5mm}}
 \multido{\i=1+1}{#4}{\includegraphics[scale=.1]{egyptian/egypt_lotus.pdf}\hspace{0.5mm}}
 \multido{\i=1+1}{#5}{\includegraphics[scale=.1]{egyptian/egypt_scroll.pdf}\hspace{0.5mm}}
 \multido{\i=1+1}{#6}{\includegraphics[scale=.1]{egyptian/egypt_heel.pdf}\hspace{0.5mm}}
 \multido{\i=1+1}{#7}{\includegraphics[scale=.1]{egyptian/egypt_stroke.pdf}\hspace{0.5mm}}
 \hspace{.5mm}
}





\title{Computing quadratures}

\begin{document}
\begin{abstract}
In this activity we will compute some basic quadratures.
\end{abstract}
\maketitle


When computing a quadrature of a shape in the method of the ancient
Greeks, one needs to produce a line segment whose length gives the
side of a square of equal area to the original shape.


\begin{question}
Consider the figure below. Explain how one could construct it and
what segment $x$ represents.
\begin{image}
\begin{tikzpicture}[geometryDiagrams]
\draw[thin] (2,0) arc (0:180:2cm);
\draw[thin] (-2,0)--(1,0);
\draw[thin] (1,0)--(2,0);
\draw[decoration={brace,mirror,raise=.2cm},decorate,thin] (-1.9,0)--(.9,0);
\draw[decoration={brace,mirror,raise=.2cm},decorate,thin] (1.1,0)--(1.9,0);
\draw (1,0)--({2*cos(60)},{2*sin(60)});
\node at (.8,{sin(60)}) {$x$};
\node at (-.5,-.5) {$n$};
\node at (1.5,-.5) {$1$};
\end{tikzpicture}
\end{image}
\end{question}


\begin{question}
Construct a rectangle whose side lengths are 8 units and 5 units.  Then construct its quadrature.  Explain your construction step-by-step, and tell why it works!
%How do you compute the quadrature of an $a\times b$ rectangle?
\end{question}

\begin{question}
Construct a triangle whose base has length 8 units and whose height has length 5 units.  Then construct its quadrature.  Explain your construction step-by-step, and tell why it works!
%How do you compute the quadrature of a triangle?
\end{question}


\begin{question}
Suppose you have a square whose side length is 8 units and another square whose side length is 15 units.  How would you construct the quadrature of the two areas together?  Explain how you know.
%How do you compute the quadrature of several squares?
\end{question}


\begin{question}
How do you compute the quadrature of a polygon?
\end{question}
\end{document}
