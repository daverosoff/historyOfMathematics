\documentclass[nooutcomes]{ximera}

\title{Begin to Count}

\begin{document}
\begin{abstract}
    Activity: Bunt Chapter 1, Part 1
\end{abstract}
\maketitle

\begin{problem}
Write the numbers from 1 to 100 (counting by fives) in hieroglyphic.
\end{problem}

\begin{problem}
Compare and contrast the hieroglyphic counting system with our (Hindu-Arabic) system.
\end{problem}

\begin{problem}
Perform the following operations in hieroglyphic, using the method that ancient Egyptians would have:
\begin{enumerate}
    \item $2371 + 185$
    \item $3914 - 1609$
\end{enumerate}
\end{problem}

\begin{problem}
Write $\frac{4}{17}$ as a sum of unique unit fractions in both Hindu-Arabic symbols and in hieroglyphic.
\end{problem}

\begin{problem}
Solve the following using the ``doubling and adding'' method of multiplication.  You may use Hindu-Arabic numerals if you'd like!
\begin{enumerate}
    \item $13 \times 33$
    \item $36 \div 5$
    \item $6 \div 17$
\end{enumerate}
\end{problem}

\begin{problem}
What is the value of the tables that ancient Egyptians often used for their calculations?
\end{problem}


\end{document}