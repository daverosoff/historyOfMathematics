\documentclass{ximera}

\graphicspath{{./}{thePythagoreanTheorem/}{deMoivreSavesTheDay/}{complexNumbersFromDifferentAngles/}{trianglesOnACone/}{cityGeometry/}{EuclidAndGeometry/}}

\usepackage{gensymb}
\usepackage[margin=1in]{geometry}

%\usepackage{hyperref}


\usepackage{tikz}
\usepackage{tkz-euclide}
\usetkzobj{all}
\tikzstyle geometryDiagrams=[ultra thick,color=blue!50!black]
\newcommand{\tri}{\triangle}
\renewcommand{\l}{\ell}
\renewcommand{\P}{\mathcal{P}}
\newcommand{\R}{\mathbb{R}}
\newcommand{\Q}{\mathbb{Q}}

\newcommand{\Z}{\mathbb Z}
\newcommand{\N}{\mathbb N}
\newcommand{\ph}{\varphi}

\renewcommand{\vec}{\mathbf}
\renewcommand{\d}{\,d}



%% Egyptian symbols

\usepackage{multido}
\newcommand{\egmil}[1]{\multido{\i=1+1}{#1}{\includegraphics[scale=.1]{egyptian/egypt_person.pdf}\hspace{0.5mm}}}
\newcommand{\eghuntho}[1]{\multido{\i=1+1}{#1}{\includegraphics[scale=.1]{egyptian/egypt_fish.pdf}\hspace{0.5mm}}}
\newcommand{\egtentho}[1]{\multido{\i=1+1}{#1}{\includegraphics[scale=.1]{egyptian/egypt_finger.pdf}\hspace{0.5mm}}}
\newcommand{\egtho}[1]{\multido{\i=1+1}{#1}{\includegraphics[scale=.1]{egyptian/egypt_lotus.pdf}\hspace{0.5mm}}}
\newcommand{\eghun}[1]{\multido{\i=1+1}{#1}{\includegraphics[scale=.1]{egyptian/egypt_scroll.pdf}\hspace{0.5mm}}}
\newcommand{\egten}[1]{\multido{\i=1+1}{#1}{\includegraphics[scale=.1]{egyptian/egypt_heel.pdf}\hspace{0.5mm}}}
\newcommand{\egone}[1]{\multido{\i=1+1}{#1}{\includegraphics[scale=.1]{egyptian/egypt_stroke.pdf}\hspace{0.5mm}}}
\newcommand{\egyptify}[7]{
 \multido{\i=1+1}{#1}{\includegraphics[scale=.1]{egyptian/egypt_person.pdf}\hspace{0.5mm}}
 \multido{\i=1+1}{#2}{\includegraphics[scale=.1]{egyptian/egypt_fish.pdf}\hspace{0.5mm}}
 \multido{\i=1+1}{#3}{\includegraphics[scale=.1]{egyptian/egypt_finger.pdf}\hspace{0.5mm}}
 \multido{\i=1+1}{#4}{\includegraphics[scale=.1]{egyptian/egypt_lotus.pdf}\hspace{0.5mm}}
 \multido{\i=1+1}{#5}{\includegraphics[scale=.1]{egyptian/egypt_scroll.pdf}\hspace{0.5mm}}
 \multido{\i=1+1}{#6}{\includegraphics[scale=.1]{egyptian/egypt_heel.pdf}\hspace{0.5mm}}
 \multido{\i=1+1}{#7}{\includegraphics[scale=.1]{egyptian/egypt_stroke.pdf}\hspace{0.5mm}}
 \hspace{.5mm}
}




%\prerequisites{algebra}
%\outcomes{methodOfFalsePosition}

\title{The method of false position}
\begin{document}
\begin{abstract}
In this activity we will seek to understand the method of false position.\
\end{abstract}
\maketitle


\begin{exercise}
Solve the following algebra problem:
\begin{align*}
x^2 + y^2 &= 52\\
2x&=3y
\end{align*}
\end{exercise}


\begin{question}
While I am not sure which method you used to solve this problem, ancient Egyptians used the method of \textit{false position} to solve problems like this. Moreover, such a method was taught in American schools until the mid 1800's. Here is the solution using false position---without any explanation!
\begin{enumerate}
\item Set $x=3$ and $y=2$.
\item $3^2 + 2^2 = 13$.
\item $52/13 = 4$.
\item $\sqrt{4} = 2$.
\item $x = 2\cdot 3$ and $y = 2\cdot 2$.
\end{enumerate}
Explain the algorithm used and give another example to show you know how it
is done.
\end{question}


\begin{exploration}
Can you explain \textbf{why} the method of false position works?
\end{exploration}

\begin{exercise}
Solve the following problem: One hundred dollars is to be split
among four siblings: Ali, Brad, Cara, and Denise where Brad gets
four more dollars than Ali, Cara gets eight more dollars than Brad,
and Denise gets twice as much as Cara. How much does each sibling get?
\end{exercise}



\break

\begin{question}
Here is the solution by \textit{double false position}:
\begin{enumerate}
\item Suppose Ali gets $6$ dollars.
\item The total now is not $100$, but $70$. We are too low by $30$. 
\item Now suppose Ali gets $8$ dollars.
\item The total now is not $100$, but $80$. We are too low by $20$.
\item Compute
\[
\frac{8\cdot 30 - 6\cdot 20}{30-20} = 12.
\]
This is the correct answer. 
\end{enumerate}
Explain the algorithm used and give another example to show you know how it
is done.
\end{question}

\begin{exploration}
Can you explain \textbf{why} the method of double false position works?
\end{exploration}
\end{document}
