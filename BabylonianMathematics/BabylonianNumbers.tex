\documentclass[nooutcomes]{ximera}

%\prerequisites{algebra}
%\outcomes{placeValue,squareRoots}
\graphicspath{{./}{thePythagoreanTheorem/}{deMoivreSavesTheDay/}{complexNumbersFromDifferentAngles/}{trianglesOnACone/}{cityGeometry/}{EuclidAndGeometry/}}

\usepackage{gensymb}
\usepackage[margin=1in]{geometry}

%\usepackage{hyperref}


\usepackage{tikz}
\usepackage{tkz-euclide}
\usetkzobj{all}
\tikzstyle geometryDiagrams=[ultra thick,color=blue!50!black]
\newcommand{\tri}{\triangle}
\renewcommand{\l}{\ell}
\renewcommand{\P}{\mathcal{P}}
\newcommand{\R}{\mathbb{R}}
\newcommand{\Q}{\mathbb{Q}}

\newcommand{\Z}{\mathbb Z}
\newcommand{\N}{\mathbb N}
\newcommand{\ph}{\varphi}

\renewcommand{\vec}{\mathbf}
\renewcommand{\d}{\,d}



%% Egyptian symbols

\usepackage{multido}
\newcommand{\egmil}[1]{\multido{\i=1+1}{#1}{\includegraphics[scale=.1]{egyptian/egypt_person.pdf}\hspace{0.5mm}}}
\newcommand{\eghuntho}[1]{\multido{\i=1+1}{#1}{\includegraphics[scale=.1]{egyptian/egypt_fish.pdf}\hspace{0.5mm}}}
\newcommand{\egtentho}[1]{\multido{\i=1+1}{#1}{\includegraphics[scale=.1]{egyptian/egypt_finger.pdf}\hspace{0.5mm}}}
\newcommand{\egtho}[1]{\multido{\i=1+1}{#1}{\includegraphics[scale=.1]{egyptian/egypt_lotus.pdf}\hspace{0.5mm}}}
\newcommand{\eghun}[1]{\multido{\i=1+1}{#1}{\includegraphics[scale=.1]{egyptian/egypt_scroll.pdf}\hspace{0.5mm}}}
\newcommand{\egten}[1]{\multido{\i=1+1}{#1}{\includegraphics[scale=.1]{egyptian/egypt_heel.pdf}\hspace{0.5mm}}}
\newcommand{\egone}[1]{\multido{\i=1+1}{#1}{\includegraphics[scale=.1]{egyptian/egypt_stroke.pdf}\hspace{0.5mm}}}
\newcommand{\egyptify}[7]{
 \multido{\i=1+1}{#1}{\includegraphics[scale=.1]{egyptian/egypt_person.pdf}\hspace{0.5mm}}
 \multido{\i=1+1}{#2}{\includegraphics[scale=.1]{egyptian/egypt_fish.pdf}\hspace{0.5mm}}
 \multido{\i=1+1}{#3}{\includegraphics[scale=.1]{egyptian/egypt_finger.pdf}\hspace{0.5mm}}
 \multido{\i=1+1}{#4}{\includegraphics[scale=.1]{egyptian/egypt_lotus.pdf}\hspace{0.5mm}}
 \multido{\i=1+1}{#5}{\includegraphics[scale=.1]{egyptian/egypt_scroll.pdf}\hspace{0.5mm}}
 \multido{\i=1+1}{#6}{\includegraphics[scale=.1]{egyptian/egypt_heel.pdf}\hspace{0.5mm}}
 \multido{\i=1+1}{#7}{\includegraphics[scale=.1]{egyptian/egypt_stroke.pdf}\hspace{0.5mm}}
 \hspace{.5mm}
}



%\usepackage[margin=1in]{geometry}

\title{Babylonian numbers} 


\begin{document}
\begin{abstract}In this activity we explore the number system of the ancient
  Babylonians.
\end{abstract} 
\maketitle


The ancient Babylonians used cuneiform characters to write their
numbers.

\begin{exercise}
What are the 2 basic ancient Babylonian numerical symbols and what do
they mean?
\end{exercise}


%\begin{exercise}
%Express the numbers 
%\[
%1, \qquad 7,\qquad 11,\qquad 53,\qquad 101, \quad 39600
%\]
%as the ancient Babylonians would. 
%\end{exercise}


%\begin{question}
%Count from 58 to 62 using the ancient Babylonian symbols. 
%\end{question}


\begin{exploration}
Discuss the limitations of the Babylonian system. Then debate whether
these so-called limitations were actually limitations at all.
\end{exploration}

\begin{exploration}
Is the Babylonian system more of a place-value system or a
concatenation system?
\end{exploration}


\begin{problem} 
Fill out the following table, simplifying any calculations.
\[
\begin{array}{| c | c || c | c || c | c |}
\hline %\bigstrut
Hindu-Arabic  & Cuneiform & Hindu-Arabic & Cuneiform & Hindu-Arabic & Cuneiform \\ \hline\hline 
5 \times 1 &\rule[7mm]{20mm}{0mm}\hspace{20mm}  & 5 \times 2  &\hspace{20mm}  & 5 \times 3 & \hspace{20mm}\\ \hline
5 \times 4 &\rule[0mm]{0mm}{7mm}   & 5 \times 5 &  & 5 \times 6 &   \\ \hline
5 \times 7 &\rule[0mm]{0mm}{7mm}   & 5 \times 8 &  & 5 \times 9 &   \\ \hline
5 \times 10 &\rule[0mm]{0mm}{7mm}   & 5 \times 20 &  & 5 \times 30 &   \\ \hline
5 \times 40 &\rule[0mm]{0mm}{7mm}   & 5 \times 50 &  & \frac15 &   \\ \hline
\frac{1}{4} &\rule[0mm]{0mm}{7mm}   & \frac{1}{9} &  & \frac{1}{10}&   \\ \hline
\frac{5}{6} & \rule[0mm]{0mm}{7mm}  & \frac{1}{20} &  & \frac{1}{100} &   \\ \hline
\end{array}
\]
\end{problem}

%

\begin{problem}
Use your table to make the following calculations.  You should work in base sixty, though you may use Hindu-Arabic numerals.
\begin{enumerate}
    \item $34 \times 5$
    \item $1,47 \div 5$
    \item $150 \div 4$
    \item $8,6,15 \div 6,40$
\end{enumerate}
\end{problem}

\end{document}
