\documentclass{ximera}

\title{Babylonian Geometry}

\begin{document}
\begin{abstract}
\end{abstract}
\maketitle

As we began to see in the previous section's readings, the Babylonians could solve many different kinds of problems. In some ways, their methods were more advanced than the Egyptian methods.  We would like to study (or look again at) the ways the Babylonians approximated square roots, how they solved quadratic equations, and several different kinds of geometry problems.  Again, with material that is repeated in more than one place, it's good to pay attention to the different opinions of different authors!




\section{Readings}

First Reading: \link[A Babylonian Geometrical Algebra]{http://www.jstor.org.proxy.lib.ohio-state.edu/stable/2686867}

Second Reading: \link[Ancient Babylonian Astronomers Calculated Jupiter's Position from the Area Under a Time-Velocity Graph]{http://science.sciencemag.org.proxy.lib.ohio-state.edu/content/351/6272/482}

Third Reading: \link[Pythagoras's Theorem in Babylonian Mathematics]{http://www-history.mcs.st-and.ac.uk/HistTopics/Babylonian_Pythagoras.html}




\section{Questions}

\begin{question}
The time interval over which the ancient Babylonians made observations about the position of Jupiter was how many days long?  $\answer[given]{60}$ days.
\end{question}


\begin{question}
Which of the following is not a challenge when translating ancient Babylonian tablets?
\begin{multipleChoice}
\choice[correct]{We do not know the ancient Babylonian language.}
\choice{Some parts of the documents are broken off or missing.}
\choice{Scribes occasionally made errors when copying documents.}
\choice{Tables of numbers are not usually labeled with their purpose.}
\end{multipleChoice}
\end{question}



\begin{question}
What are the most important points of this reading?
\begin{freeResponse}
\end{freeResponse}
\end{question}



\end{document}