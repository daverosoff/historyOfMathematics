\documentclass[nooutcomes]{ximera}

\title{Arabic Mathematics}

\begin{document}
\begin{abstract}
    
\end{abstract}
\maketitle

Dunham leaves a pretty big gap after Chapter 5, which he covers just a little bit in the epilogue to the chapter.  During the years between Heron (we'll say after 150 AD) and Cardano (in the 1500s), people were still doing mathematics, of course!  During this period, mathematics was being studied mostly in the Middle East and India.  Our second reading helps us get a better sense of the contributions of Arabic mathematicians during this time period.


\section{Readings}
First reading: Dunham, Chapter 5, pages 121 - 132 

Second reading: \link[Arabic Mathematics: forgotten brilliance?]{http://www-history.mcs.st-and.ac.uk/HistTopics/Arabic_mathematics.html}

\section{Questions}

\begin{question}
How many of Archimedes' works were translated into Arabic? $\answer[given]{2}$
\end{question}

\begin{question}
Why did Arabic mathematicians use trigonometric tables?
\begin{multipleChoice}
\choice{For examples of algebra problems.}
\choice {To teach school children.}
\choice [correct]{In astronomy.}
\choice {In building projects.}
\end{multipleChoice}
\end{question}


\begin{question}
What are the most important points from this reading?
\begin{freeResponse}
\end{freeResponse}

\end{question}




\end{document}
