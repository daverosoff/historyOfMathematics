\documentclass[nooutcomes]{ximera}

\title{Mathematicians after Archimedes}

\begin{document}
\begin{abstract}
    
\end{abstract}
\maketitle

Archimedes was certainly a mathematical genius, and following in his footsteps would certainly have been challenging.  Our subject now is these mathematicians, living in changing times.  It is relatively fitting to the story of mathematics that Archimedes was killed by the Romans, as the Roman takeover of much of the Greek world changed the way that people thought about and did mathematics for many years.

Here are some names of mathematicians whose contributions we will discuss at least briefly in class.  By the end of our time on this chapter, you should be able to say a few words about what each person did.
\begin{itemize}
\item Eratosthenes (the subject of our second reading)
\item Apollonius
\item Heron (the subject of our first reading)
\item Ptolemy
\item Diophantus
\end{itemize}

The second reading is a little more history focused than mathematically focused.  You should make sure to first read Dunham's version of Eratosthenes' proof in the first reading.  This reading helps us to see some of the challenges that historical scholars face with mathematical works.


\section{Readings}
First reading: Dunham, Chapter 5, pages 113 - 121 

Second reading: \link[The Origin and Value of the Stadion Unit used by Eratosthenes in the Third Century B.C.]{http://www.jstor.org/stable/41133819}

\section{Questions}

\begin{question}
How Greek feet are in one stadion? $\answer[given]{600}$
\end{question}

\begin{question}
What does the author suggest was the usual precision for measuring long distances?
\begin{multipleChoice}
\choice{Round to the nearest inch.}
\choice {Round to the nearest foot.}
\choice {Round to the nearest stadion.}
\choice [correct]{Round to the nearest 10 stadia.}
\end{multipleChoice}
\end{question}


\begin{question}
What are the most important points from this reading?
\begin{freeResponse}
\end{freeResponse}

\end{question}




\end{document}
