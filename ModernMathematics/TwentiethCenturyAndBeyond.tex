\documentclass[nooutcomes]{ximera}

\title{The Twentieth Century and Beyond}

\begin{document}
\begin{abstract}
    
\end{abstract}
\maketitle

Dunham's story ends with Cantor, but the story of mathematics certainly doesn't.  Hopefully, as we've talked about the history of mathematics this semester, you've seen some themes emerging.  Mathematics is always growing and changing, and the way mathematicians think about the subject is also growing and changing.  The first reading is an attempt to add to these themes.  It is written by Michael Atiyah, a famous topologist who won the Fields Medal in 1966.  You don't need to read this entire article in detail, but be sure to read enough to get a sense of the themes that Atiyah is describing and how they are similar to (or different from) some of the themes we have discussed previously.

The Fields Medal is the highest award given to mathematicians, and is sometimes referred to as the ``Nobel Prize of Mathematics''.  The second reading is a bit of history on this prize, and the third reading a list of past winners of the prize.  It's worthwhile to note that the first woman to win the Fields Medal was Maryam Mirzakhani, who won the prize in 2014 and passed away in 2017.



\section{Readings}
First reading: \link[Mathematics In The Twentieth Century]{http://www.jstor.org/stable/2695275}

Second reading: \link[The Fields Medal]{http://www.mathunion.org/general/prizes/fields/details/}

Third reading: \link[List of Fields Medalists]{http://www.mathunion.org/general/prizes/fields/prizewinners/}



\section{Questions}

\begin{question}
The Fields Medal is awarded every $\answer[given]{4}$ years.
\end{question}

\begin{question}
Which of the following is not a theme discussed by Atiyah?
\begin{multipleChoice}
\choice{Local to Global}
\choice{Geometry versus Algebra}
\choice {Techniques in Common}
\choice [correct]{Increase in Precision}
\end{multipleChoice}
\end{question}


\begin{question}
What are the most important points from this reading?
\begin{freeResponse}
\end{freeResponse}

\end{question}




\end{document}
