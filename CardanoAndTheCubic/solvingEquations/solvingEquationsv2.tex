\documentclass[nooutcomes]{ximera}

\graphicspath{{./}{thePythagoreanTheorem/}{deMoivreSavesTheDay/}{complexNumbersFromDifferentAngles/}{trianglesOnACone/}{cityGeometry/}{EuclidAndGeometry/}}

\usepackage{gensymb}
\usepackage[margin=1in]{geometry}

%\usepackage{hyperref}


\usepackage{tikz}
\usepackage{tkz-euclide}
\usetkzobj{all}
\tikzstyle geometryDiagrams=[ultra thick,color=blue!50!black]
\newcommand{\tri}{\triangle}
\renewcommand{\l}{\ell}
\renewcommand{\P}{\mathcal{P}}
\newcommand{\R}{\mathbb{R}}
\newcommand{\Q}{\mathbb{Q}}

\newcommand{\Z}{\mathbb Z}
\newcommand{\N}{\mathbb N}
\newcommand{\ph}{\varphi}

\renewcommand{\vec}{\mathbf}
\renewcommand{\d}{\,d}



%% Egyptian symbols

\usepackage{multido}
\newcommand{\egmil}[1]{\multido{\i=1+1}{#1}{\includegraphics[scale=.1]{egyptian/egypt_person.pdf}\hspace{0.5mm}}}
\newcommand{\eghuntho}[1]{\multido{\i=1+1}{#1}{\includegraphics[scale=.1]{egyptian/egypt_fish.pdf}\hspace{0.5mm}}}
\newcommand{\egtentho}[1]{\multido{\i=1+1}{#1}{\includegraphics[scale=.1]{egyptian/egypt_finger.pdf}\hspace{0.5mm}}}
\newcommand{\egtho}[1]{\multido{\i=1+1}{#1}{\includegraphics[scale=.1]{egyptian/egypt_lotus.pdf}\hspace{0.5mm}}}
\newcommand{\eghun}[1]{\multido{\i=1+1}{#1}{\includegraphics[scale=.1]{egyptian/egypt_scroll.pdf}\hspace{0.5mm}}}
\newcommand{\egten}[1]{\multido{\i=1+1}{#1}{\includegraphics[scale=.1]{egyptian/egypt_heel.pdf}\hspace{0.5mm}}}
\newcommand{\egone}[1]{\multido{\i=1+1}{#1}{\includegraphics[scale=.1]{egyptian/egypt_stroke.pdf}\hspace{0.5mm}}}
\newcommand{\egyptify}[7]{
 \multido{\i=1+1}{#1}{\includegraphics[scale=.1]{egyptian/egypt_person.pdf}\hspace{0.5mm}}
 \multido{\i=1+1}{#2}{\includegraphics[scale=.1]{egyptian/egypt_fish.pdf}\hspace{0.5mm}}
 \multido{\i=1+1}{#3}{\includegraphics[scale=.1]{egyptian/egypt_finger.pdf}\hspace{0.5mm}}
 \multido{\i=1+1}{#4}{\includegraphics[scale=.1]{egyptian/egypt_lotus.pdf}\hspace{0.5mm}}
 \multido{\i=1+1}{#5}{\includegraphics[scale=.1]{egyptian/egypt_scroll.pdf}\hspace{0.5mm}}
 \multido{\i=1+1}{#6}{\includegraphics[scale=.1]{egyptian/egypt_heel.pdf}\hspace{0.5mm}}
 \multido{\i=1+1}{#7}{\includegraphics[scale=.1]{egyptian/egypt_stroke.pdf}\hspace{0.5mm}}
 \hspace{.5mm}
}




\title{Solving equations}
\begin{document}
\begin{abstract}
In this activity we will solve second and third degree equations.
\end{abstract}
\maketitle


Finding roots of quadratic polynomials is somewhat complex. We want to
find $x$ such that
\[
ax^2 + bx + c = 0.
\]
I know you already know how to do this. However, pretend for a moment
that you don't. This would be a really hard problem. We have evidence
that it took humans around 1000 years to solve this problem in
generality, with the first general solutions appearing in Babylon and China
around 2500 years ago. Let's begin with an easier problem: make $a=1$ and try to
solve $x^2 + bx = c$.
 

\begin{problem} Geometrically, you could visualize $x^2 + bx = c$ as an $x \times x$ square
along with a $b\times x$ rectangle. Make a blob for $c$ on the other side.  Draw a picture of this!
\end{problem}


\begin{question} What is the total area of the shapes in your picture?
\end{question}




\begin{problem} Now draw a new picture: take your $b\times x$ rectangle and divide it into two $(b/2)\times x$ rectangles.
\end{problem}


\begin{question} What is the total area of the shapes in your picture?
\end{question}

\begin{problem} Draw a next picture in your sequence: take both of your $(b/2)\times x$ rectangles and snuggie them
next to your $x\times x$ square on adjacent sides. You should now have
what looks like an $(x + \frac{b}{2}) \times (x +
\frac{b}{2})$ square with a corner cut out of it.
\end{problem}


\begin{question} What is the total area of the shapes in your picture?
\end{question}


Finally, your big $(x + \frac{b}{2}) \times (x + \frac{b}{2})$ has a
piece missing, a $(b/2) \times (b/2)$ square, right? So if you add
that piece in on both sides, the area of both sides of your picture
had better be $c + (b/2)^2$. From your picture you will find that:
\[
\left(x + \frac{b}{2}\right)^2 = c + \left(\frac{b}{2}\right)^2
\]

\begin{question} 
Can you find $x$ at this point?
\end{question}


\begin{question}
Explain how to solve $ax^2 + bx + c = 0$.
\end{question}



\subsection*{Cubic Equations}

While the quadratic formula was discovered around 2500 years ago,
cubic equations proved to be a tougher nut to crack. A general
solution to a cubic equation was not found until the 1500's - and under some pretty strange circumstances!  See your text for the low-down on all of the drama.

We'll show you the Ferro-Tartaglia method\index{Ferro-Tartaglia
  method} for finding at least one root of a cubic of the form
\[
x^3+ px + q.
\]

All I can tell you are these three steps:
\begin{enumerate}
\item Replace $x$ with $u+v$. 
\item Set $uv$ so that all of the terms are eliminated except for $u^3$,
$v^3$, and constant terms.  
\item Clear denominators and use the quadratic formula.
\end{enumerate}



\begin{question} How many solutions are we supposed to have in total?
\end{question}

\begin{question}
Use the Ferro-Tartaglia method to solve $x^3 + 9x -26 =0$.
\end{question}

\begin{question}
How many solutions should our equation above have? Where/what are they?
Hint: Make use of an old forgotten foe\dots
\end{question}

\begin{question}
Is the method described here the same as the one in our text as the proof of the ``great theorem''?  Explain why or why not.
\end{question}

\begin{question}
Use the Ferro-Tartaglia method to solve $x^3 = 15x + 4$.  What do you notice?
\end{question}



\begin{question} How do we do this procedure for other equations of the form
\[
x^3 + px + q = 0?
\]
Give an algebraic formula as your solution.
\end{question}



\begin{question}
Would Ferro, Tartaglia, Cardano, or Ferrari have answered the previous question differently than you might?
\end{question}


\end{document}